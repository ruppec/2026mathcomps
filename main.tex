\documentclass{article}
\usepackage{amsmath, amssymb, amsthm}
\usepackage{geometry}
\usepackage{graphicx}
\usepackage{hyperref}
\usepackage{xcolor}

\geometry{margin=1in}

\hypersetup{colorlinks=true, linkcolor=blue, citecolor=blue}


\newcommand{\jrs}[1]{\textcolor{blue}{JS: #1}}



\begin{document}

\title{Interactive Visualizations of Projective Varieties in Spherical Geometry}
\author{Katie Hess, Charlie Ruppe, Jake Schaefer }
\date{\today}



\maketitle



\begin{abstract}
Many mathematical theorems are clarified by thinking in projective space, where so-called ``points at infinity” are added to complete the geometry. Often these points at infinity are left out of drawings, with their relation to the other points and to each other suggested rather than directly illustrated. Yet to an observer living inside of projective space, these points at infinity would be geometrically indistinguishable from any other point. In this work, we present a real-time rendering framework for the intrinsic visualization of 3D real projective space (RP3) by ray marching directly on the 3-sphere. Leveraging this technique, we observe several properties that are most elegantly stated for projective space. We demonstrate our library's capabilities through visualizing the geometry of symmetric cubic surfaces and families of cubic curves on quartic surfaces, as well as how they manifest in this unbiased, personified view. 
\end{abstract}


\section{Introduction} 

The notion of creating a deductive formalization of mathematical properties through visualizations and intuitions of geometry dates back to ancient Greece. Notably, in his 299 B.C. book, \textit{The Elements}, Euclid posited as an axiom that parallel lines do not meet. In a painitng, lines that might be parallel in the real world appear to meet at a vanishing point.

Projective geometry provides a natural and conceptually complete setting for many geometric theorems. Statements about parallel lines, conic sections, and algebraic varieties often become simpler and more unified when expressed projectively. However, standard visualizations often treat points at infinity as artifacts rather than intrinsic components of the geometry. 



To an observer living inside projective space, no point would be distinguished as infinite. The geometry is homogeneous, as every point admits identical local structure, and there is no intrinsic notion of boundary. The apparent singularity of infinity arises only from extrinsic representations. In this work, we develop a rendering framework that allows the viewer to inhabit $\mathbb{RP}^3$ directly, removing the artificial distinction between finite and infinite points.
To accomplish this, we use the identification $\mathbb{RP}^3 \cong S^3 / \{\pm 1\},$
and view real projective 3-space as the antipodal quotient of the 3-sphere. Rather than projecting projective space into Euclidean coordinates, we perform ray marching directly along geodesics of $S^3$ and preserve the intrinsic geometry. Surfaces defined by homogeneous equations are evaluated on the sphere, and antipodal symmetry ensures the correct projective identification.

In Sec. \ref{sec:background}, we review the structure of $\mathbb{RP}^3$ and its realization as the antipodal quotient of $S^3$. Sec. \ref{sec:framework} describes our intrinsic ray-marching framework and numerical methods. Our implementation is outlined in \ref{sec:implementation}. In Sec. \ref{sec:examples} we present examples of projective varieties visualized using this approach. We conclude with a discussion of implications and future directions.



%\begin{itemize}
    %\item motivation 
    %\item why points at infinity are conceptually misleading in standard visualization
    %\item intrinsic visualization of RP3
    %\item description of contributions
%\end{itemize}


\section{Mathematical Background}
\label{sec:background}

\subsection{Real Projective 3-Space}

The geometric space under consideration in this work is real projective 3-space, denoted $\mathbb{RP}^3$. Formally, it is defined as the space of all one-dimensional linear subspaces of $\mathbb{R}^4$, or, equivalently,
\[
\mathbb{RP}^3 = (\mathbb{R}^4 \setminus \{0\}) / \sim,
\]
where $x \sim \lambda x$ for every nonzero scalar $\lambda \in \mathbb{R}$. A point in $\mathbb{RP}^3$ therefore represents an entire line through the origin in $\mathbb{R}^4$. Using homogeneous coordinates, we write such a point as
\[
[x_0 : x_1 : x_2 : x_3] := [\lambda x_0 : \lambda x_1 : \lambda x_2 : \lambda x_3]
\quad \text{for all } \lambda \neq 0.
\]
Projective space may additionally be understood as a completion of Euclidean space. In fact, we recover standard Euclidean coordinates via
\[
[x_0 : x_1 : x_2 : x_3] \mapsto\left(\frac{x_1}{x_0},\, \frac{x_2}{x_0},\, \frac{x_3}{x_0}\right),
\]
where we have identified $\mathbb{R}^3$ with the affine chart $[x_0 : x_1 : x_2 : x_3]$, such that $x_0 \neq 0$. The remaining points, i.e., those with $x_0 = 0$, are often described as ``points at infinity." In this interpretation, parallel lines in $\mathbb{R}^3$ intersect at a unique point at infinity determined by their direction, similar to how railroad tracks vanish at a different point in the horizon depending on what angle they take with your vision.

This language of infinity can be misleading. The distinction between finite and infinite points depends entirely on the choice of affine chart. Changing coordinates moves the hyperplane $x_0 = 0$ to another location, altering which points are labeled infinite. From the intrinsic perspective of $\mathbb{RP}^3$, there is no preferred hyperplane and therefore no distinguished set of infinite points. Every point possesses identical local geometric structure.

Consider the identification
\begin{equation}
    \mathbb{RP}^3 \cong S^3 / \{\pm 1\},\quad\text{where }S^3 = \{ x \in \mathbb{R}^4 : \|x\| = 1 \}.
\end{equation}
Here, $S^3$ is the unit 3-sphere. Note that each line through the origin intersects $S^3$ in exactly two antipodal points. Identifying these antipodal pairs of intersection points produces $\mathbb{RP}^3$. Thus, projective space inherits the smooth compact structure of the sphere modulo symmetry. 

Rather than embedding $\mathbb{RP}^3$ into Euclidean space and treating infinity as limiting, we perform geometric computations directly on $S^3$ and enforce antipodal identification. Thus, surfaces and intersections that would appear to escape to infinity in Euclidean coordinates remain entirely visible and continuous within the projective model, and the geometry becomes globally coherent.


%\begin{itemize}
    %\item define rp3 
    %\item discuss points at infinity and projective completion
    %\item intrinsic visualization of RP3
    %\item explain why no point is geometrically distinguished
%\end{itemize}

\subsection{Spherical Geometry}

In Euclidean space, the paths that minimize distance, or the geodesics, are the straight ones: straight lines extend infinitely and distances are measured with respect to a flat metric. On the sphere, however, the natural notion of a straight line is a great circle geodesic, a curve that locally minimizes distance while remaining on the surface. These great circles are compact and obtained by intersecting the sphere with two-dimensional places through the origin in $\mathbb{R}^4$. 

To an observer living in $S^3$, these geodesics would appear to be the straightest possible paths available, even though they appear curved when viewed extrinsically. Were they to ride a bike on this surface, traveling along a great circle would not require any turning of the handle bars. This distinction is essential to our visualizations, as we do not treat the sphere as a curved object sitting inside a higher-dimensional flat space. Instead, we treat the sphere as the ambient space itself.

Geometrically, $S^3$ has constant positive curvature. This curvature explains several features, including the tendency for geodesics to bend towards one another, that triangles have angle sums greater than pi, and the compactness of the global topology. This compactness gives us a powerful result about the space: rays \jrs{(specify that they follow geodesics?)} never diverge or escape. Projective phenomena that appear unbounded in Euclidean coordinates become fully contained when viewed on $S^3$.

Because $S^3$ is compact and geodesics are periodic, rays don't escape to infinity. 


After identifying antipodal points to obtain $\mathbb{RP}^3$, the space remains compact but becomes non-simply connected. 



\begin{itemize}
    \item explain intrinsic viewpoint
    \item contrast with Euclidean intuition
    \item short discussion of curvature and topology
\end{itemize}

\section{Rendering Framework}
\label{sec:framework}

\subsection{Ray Marching on $S^3$}

In Euclidean ray tracing, rays follow the straight line geodesics which originate at a camera position and travel outward until they intersect a surface. Implicitly, the trajectory of these rays is governed by the flat Euclidean metric. In our setting, we treat $S^3$ as the ambient space itself, and thus the rays should instead follow geodesics of $S^3$.

Given a point $p \in S^3$ and a unit tangent vector $v \in T_p S^3$ with $\langle p, v \rangle = 0$, the geodesic through $p$ in direction $v$ is given by
\begin{equation}
    \gamma(t) = \cos(t)\, p + \sin(t)\, v.
\end{equation}
Since $\|\gamma(t)\|=1$, this curve remains on $S^3$ for all $t$ and traces out a great circle. Rendering, therefore, is simply evaluating surfaces along such geodesics. However, rather than solving for the first intersection of a geodesic with a surface as we would with ray tracing, we instead utilize ray marching. Here, we incrementally step forward along $\gamma(t)$ and test for intersections. 

We concern ourselves with surfaces specified by homogeneous polynomials $F:\mathbb{R}^4\to\mathbb{R}$, where the level sets $F(x_0,x_1,x_2,x_3) = 0$ define how the surfaces present themselves in our space \jrs{(change i dont like wording i think)}. Detecting an intersection is just a root-finding problem for the single-variable function
\begin{equation}
    f(t) = F(\gamma(t)).
\end{equation}
We march forward in the parameter $t$ with step size $\Delta t$, and a sign change in $f(t)$ from step $t$ to $t+\Delta t$ indicates that the surface has been crossed. We then refine the intersection points using a bracketing method. We approximate the restriction of $F$ along the ray using the Bernstein basis, which provides stable bounds and allows us to guarantee convergence within a chosen tolerance. \jrs{idk anything about Bernstein}

Great circles on $S^3$ satisfy the periodicity constraint $\gamma(t)=\gamma(2\pi+t)$. In addition, under the aforementioned antipodal quotient $\mathbb{RP}^3 \cong S^3/\{\pm 1\}$, we have that $[\gamma(t)]=[\gamma(\pi+t)]$. Thus, it suffices to impose a maximum parameter length of $\pi$ to avoid the ray redundantly traversing the space. 


After determining the first intersection point $x_\text{hit}:=\gamma(t_\text{hit})$, we compute the surface normals intrinsically by projecting the Euclidean gradient $\nabla F|_{x_\text{hit}}\in\mathbb{R}^4$ onto the tangent space of $S^3$ \jrs{(equationhere?)}. The resulting normal vector is used for shading and lighting calculations.


We also incorporate a path tracing option to model lighting. Secondary rays are generated at intersection points to simulate reflection and global illumination effects. These rays again follow geodesics in $S^3$, ensuring that lighting computations are consistent with the intrinsic geometry. This enhances depth perception and helps communicate accurate geometric structure.


maybe not necessary?
\begin{itemize}
    \item explain why homogeneous formulation is natural
    \item discuss normalization
\end{itemize}



\subsection{Bernstein Basis Approximation}

To compute ray surface intersections reliably, we approximate certain functions using the Bernstein basis. For a polynomial of degree $n$ in one variable on the interval $[0,1]$, the Bernstein basis consists of the functions
\[
B_{k}^{n}(t) = \binom{n}{k} t^{k}(1-t)^{n-k}, 
\quad k = 0, \dots, n.
\]
Any polynomial $p(t)$ of degree at most $n$ can be written uniquely as
\[
p(t) = \sum_{k=0}^{n} c_k B_k^n(t).
\]

One important benefit of using the Bernstein basis is that it allows for control over the range of the function. On $[0,1]$, the polynomial $p(t)$ is within the convex hull of its coefficients $\{c_k\}$. The minimum and maximum values of $p(t)$ are bounded by the minimum and maximum of the coefficients. This property makes the Bernstein form numerically stable and well-suited for root-finding, since it allows us to find whether a function can change sign on a given interval.

Intersection detection reduces to finding roots of a function
\[
f(t) = F(\gamma(t)),
\]
where $\gamma(t)$ is a geodesic on $S^3$. After restricting to a bounded parameter interval, we approximate $f(t)$ in Bernstein form. If all Bernstein coefficients have the same sign, then no root occurs in that interval. If the coefficients change sign, we subdivide the interval and repeat the test.

Therefore the Bernstein basis helps us avoid unstable cancellation effects that can arise and allows us to certify the presence or absence of roots with a specific tolerance. It enables accurate rendering without requiring closed-form intersection formulas.


\section{Implementation}
\label{sec:implementation}

\subsection{WebGPU Architecture}

Our implementation uses the the \textsc{wgpu} crate which is a cross-platform graphics API patterned off of the WebGPU standard, written in pure Rust. This allows our code to compile to DirectX12 for Windows, Metal for iOS and MacOS, Vulkan for Windows Linux and Android, for WebGL and WebGPU for the web by compiling to WebAssembly. The tested platforms are Vulkan on Linux, Metal on MacOS, and WebGL for the web.
We use WGSL for our shading language.

The main advantage of using \textsc{wgpu} is the cross-platform support and the memory-safety guarantees offered by Rust.
The WebGPU standard as implemented gives us enough control over the rendering pipeline to be efficient while abstracting way some of the unnecessary and less-portable features of Vulkan.
Also, the option of using WebGPU allows for the possibility of using compute shaders or general-purpose storage buffers in future iterations of the software.

\subsection{Lowering... (TBD)}
We additionally provide an implementation of a symbolic polynomial library in Rust, which is used to formulate implicit surfaces and perform polynomial arithmetic, such as the symbolic calculation of derivatives, on the CPU. These expressions are then expanded into monomial terms using a user-specified ordering and baked into WGSL code for use as SDFs on the GPU.




\section{Examples}
\label{sec:examples}

Discuss examples we have rendered and their significance
\begin{itemize}
    \item (Jake) Bezout's theorem, group law on elliptic curves, existence theorem
    \begin{itemize}
        \item visualize a projection from a higher affine space onto our space, and compare the image of a variety defined by an ideal in that higher space and its image versus the variety defined by the intersection of the defining ideal and our space (the zariski closure)
    \end{itemize}
\end{itemize}

\subsection{Symmetric Cubic Surfaces}


\subsection{Cubic Curves on Quartic Surfaces}
The shader implements an intrinsic numerical scheme for computing intersections between geodesics in $S^3$ and homogeneous quartic hypersurfaces. Let
\[
F : \mathbb{R}^4 \to \mathbb{R}
\]
be a homogeneous polynomial of degree four. The zero set of $F$ determines a quartic hypersurface in $\mathbb{R}^4$, and its intersection with $S^3$ brings it down to a quartic surface in $\mathbb{RP}^3$.

Quartic surfaces are constructed in one implementation by


and in another by combining two homogeneous cubic polynomials that define elliptic curves. Let
\[
C_1(x_0,x_1,x_2,x_3), \qquad 
C_2(x_0,x_1,x_2,x_3)
\]
be homogeneous cubic polynomials whose vanishing loci define projective elliptic curves. From these, one forms a quartic polynomial by a homogeneous combination such as
\[
F = C_1 \cdot L_1 + C_2 \cdot L_2,
\]
where $L_1$ and $L_2$ are chosen so that $F$ has degree four.



\subsection{Bezout's Theorem}

visualize a projection from a higher affine space onto our space, and compare the image of a variety defined by an ideal in that higher space and its image versus the variety defined by the intersection of the defining ideal and our space (the zariski closure)

\section{Discussion and Conclusion}

-summarize contributions
-educational implications?
-future directions 


\bigskip

\bibliographystyle{plain}
\bibliography{references}

thurston paper

On Maximal Homogeneous 3-Geometries and Their Visualization
Emil molnar

Projective geometry and duality for graphics, games and visualization
Vaclav Skala


\end{document}
